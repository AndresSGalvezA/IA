\documentclass[]{scrreprt}

% Title Page
\title{Agentes inteligentes}
\author{Andres Sebastian Gálvez Arriaza}
\date{25 de enero de 2021}

\begin{document}
\maketitle
\section{Tron}
El mundo físico y el mundo digital están separados por algo más que la imaginación. Al menos, así ha sido hasta ahora, una época en la que la tecnología tiene un auge que muchos no habrían creído posible. En la película estadounidense Tron de 1982, este límite desaparece al incluir en materia el mundo físico en el mundo virtual. De alguna manera, ambas dimensiones existen en una sola y la única forma de escapar es comprender el funcionamiento de este sistema. 
\newline
\newline
La película muestra una historia en la que se observa un hipotético funcionamiento de una sociedad de programas dentro de una computadora. Sociedad gobernada por el Programa de Control Maestro (PCM), una inteligencia artificial creada por el director general, Dillinger, que poco a poco capaz de entablar conversaciones como lo haría una persona, con procesamiento de lenguaje natural. Entre otras cosas, también se presentan una serie de funcionalidades futuristas, tal como el rayo capaz de digitalizar la estructura molecular, cuya existencia permitiría desplazar materia, sin enfatizar en su tipo, a cualquier lugar en cuestión de segundos. Esta película, junto a su buen nivel de críticas recibidas, se convierte en una fuente de sapiencia en el mundo de la informática. El mismo director de Pixar afirmó en su momento que Tron "le inspiró a interesarse en el desarrollo de personajes mediante computadora" y que "si no la hubiese visto, películas como Toy Story no existirían".
\newline
\newline
A la fecha de la publicación de este artículo, no existe de manera pública una tecnología como la mostrada en la película. Sin embargo, la idea de tener un programa capaz de controlarlo todo es, cuando menos, complejo. Confiarle todo a un programa, dejando de lado o minimizando por completo la interacción humana, conlleva implicaciones en diversos ámbitos. Por un lado, la cuestión de "reemplazar a los humanos" aqueja a la idea del desarrollo de nuevas tecnologías. Una parte por el miedo a lo desconocido y otra muy diferente al peligro real que pueda suponer. Plantearse la interrogante acerca de qué tan prescindibles somos como especie y, además, qué tan capaces seremos de controlar una inteligencia "ajena" a la humana, no hace más que originar debates éticos con bastante fundamento. Esto desemboca en consecuencias que competen al área legal, ya que cualquier tipo de norma de conducta, sobre lo permitido y lo prohibido, es meramente una invención de humanos para humanos. Debiese ser, quizá, necesaria la creación de leyes para máquinas, que les indiquen de manera explícita qué hacer y qué no, definiendo límites a su capacidad. Tal caso es el del PCM, quien menciona en un fragmento su intención de tomar datos de la oficina del Pentágono estadounidense sin ver las posibles consecuencias. A su vez, es menester aproximarse al área tecnológica. Crear una tecnología de tal magnitud supone tener la capacidad de almacener volúmenes muy grandes de información y, con esto, la capacidad de gestionarla con aguda precisión, a tal punto de simular un cerebro humano, capaz de tomar decisiones según una serie de situaciones, previstas e imprevistas, analizar su entorno y mejorarse a sí a través de su existencia. El futuro dirá si quedará en un sueño o si, por el contrario, estas entidades serán capaces de tomar consciencia sobre su existencia y llegar a imitarnos, igualarnos y superarnos como especie.

\subsection{Agente inteligente}

En la película es posible identificar el PCM como un agente inteligente. Esta entidad presenta las cuatro propiedades clásicas que lo definen como tal:

\begin{enumerate}
    \item Interactua con un entorno: donde conviven personas que proveen entradas para el sistema y demás programas creados por los usuarios.
    \item Usa sensores para percibir el estado: capaz de procesar lenguaje natural y establecer posiciones en el espacio.
    \item Usa acciones para afectar el estado: intento de frenar el acceso a los datos de los juegos de Flynn e incluso inhibiendo distintos programas.
    \item La interacción es definida por una política de control: Control maestro, el nombre de este agente.
\end{enumerate}

Este agente se desenvuelve en un entorno, el cual se clasifica de la siguiente manera:
\begin{enumerate}
\item Parcialmente observable: no hay un control total sobre lo que sucede y las posibilidades de acción no se limitan a un grupo concreto.
\item Memoria: tomando en cuenta que el PCM conocía sobre las intenciones de vulnerarle.
\item Estocástico, dando probabilidades de ocurrencia de ciertos sucesos y realizando cálculos con ellas.
\item Continuo: donde existen intervalos de posibilidades infinitas de acciones, sujetas tanto a los otros programas como a los mismos usuarios.
\item Benigno: No se busca (de forma natural) el malestar del agente.
\end{enumerate}  
Realizar este tipo de programas requiere una combinación de ramas de la inteligencia artificial, al menos, de manera hipotética. El PCM debió requerir implementaciones de robótica, demostrando la capacidad de mover piezas articuladas; strong AI o inteligencia artificial general; razonamiento y toma de decisiones, habilitándolo para inferir sobre las acciones de Flynn, lo cual incluye visión por computador para poder localizarlo en espacio; aprendizaje de máquina, confesando este mismo que ha mejorado desde la fecha de su creación al adquierir información constante y procesamiento de lenguajes para tener la capacidad de comunicarse con eficacia con humanos.
\newline
\newline
Las aplicaciones de la inteligencia artificial están cambiando de forma importante la naturaleza de muchos procesos. Las computadoras ya no son solamente herramientas de ayuda, sino que empiezan a ser agentes creativos, agentes inteligentes, que prometen una revolución no solo en el ámbito tecnológico, sino que en un campo multidisciplinario donde deben existir tanto personas como máquinas.
\end{document}